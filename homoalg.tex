%% Basierend auf einer TeXnicCenter-Vorlage von Mark Müller
%%%%%%%%%%%%%%%%%%%%%%%%%%%%%%%%%%%%%%%%%%%%%%%%%%%%%%%%%%%%%%%%%%%%%%%

% Wählen Sie die Optionen aus, indem Sie % vor der Option entfernen  
% Dokumentation des KOMA-Script-Packets: scrguide

%%%%%%%%%%%%%%%%%%%%%%%%%%%%%%%%%%%%%%%%%%%%%%%%%%%%%%%%%%%%%%%%%%%%%%%
%% Optionen zum Layout des Artikels                                  %%
%%%%%%%%%%%%%%%%%%%%%%%%%%%%%%%%%%%%%%%%%%%%%%%%%%%%%%%%%%%%%%%%%%%%%%%
\documentclass[%
%a5paper,             % alle weiteren Papierformat einstellbar
%landscape,           % Querformat
%10pt,                % Schriftgröße (12pt, 11pt (Standard))
%BCOR1cm,             % Bindekorrektur, bspw. 1 cm
%DIVcalc,             % führt die Satzspiegelberechnung neu aus
%                       s. scrguide 2.4
%twoside,             % Doppelseiten
%twocolumn,           % zweispaltiger Satz
halfparskip*,       % Absatzformatierung s. scrguide 3.1
%headsepline,         % Trennline zum Seitenkopf  
%footsepline,         % Trennline zum Seitenfuß
titlepage,            % Titelei auf eigener Seite
%normalheadings,      % Überschriften etwas kleiner (smallheadings)
%idxtotoc,            % Index im Inhaltsverzeichnis
%liststotoc,          % Abb.- und Tab.verzeichnis im Inhalt
bibtotoc,           % Literaturverzeichnis im Inhalt
%abstracton,          % Überschrift über der Zusammenfassung an 
%leqno,               % Nummerierung von Gleichungen links
%fleqn,               % Ausgabe von Gleichungen linksbündig
%draft                % überlangen Zeilen in Ausgabe gekennzeichnet
DIV = 15,
headsepline,
openany,
BCOR=0.8cm,
pointlessnumbers,        %keine Punkte nach Überschriften
chapterprefix=true
]
{scrbook}



%% Deutsche Anpassungen %%%%%%%%%%%%%%%%%%%%%%%%%%%%%%%%%%%%%

\usepackage[ngerman]{babel}
\usepackage[T1]{fontenc}
\usepackage[utf8]{inputenc}


\usepackage{lmodern} %Type1-Schriftart für nicht-englische Texte

\usepackage{amsmath,amssymb,MnSymbol}
\usepackage{xcolor}

\usepackage{array}

\usepackage[inline]{enumitem}
\setlist[enumerate]{label=(\arabic*)}


%% Packages für Grafiken & Abbildungen %%%%%%%%%%%%%%%%%%%%%%
\usepackage{graphicx} %%Zum Laden von Grafiken
%\usepackage{subfig} %%Teilabbildungen in einer Abbildung
\usepackage{calc}

\usepackage{tikz}
\usetikzlibrary{calc,positioning,backgrounds}
\pgfdeclarelayer{background}
\pgfdeclarelayer{foreground}
\pgfsetlayers{background,main,foreground}
\usepackage{tikzpagenodes}
\usepackage{tikz-cd} 

\tikzstyle{dotnode} = [circle, fill=orange, inner sep=0, outer sep=0pt,
  minimum width=5pt]
\tikzstyle{line normal} = [draw=gray!80, line width=1pt]
\tikzstyle{filled} = [fill=orange!15]



\usepackage[colorlinks=false, pdfborder={0 0 0}]{hyperref}
\usepackage[nameinlink,german]{cleveref}

\usepackage{listings}
\usepackage[automark]{scrpage2} % Headline styles 


%% Listings setup %%%%%%%%%%%%%%%%
\lstset{
  mathescape = true,
  basicstyle = \small\normalfont\sffamily,
  frame = tb,
  framexleftmargin = 15pt,
% numbers = left,
  numberstyle = \tiny,
% numbersep = 5pt,
  breaklines = true,
  xleftmargin = 0.1\linewidth,
  xrightmargin = 0.1\linewidth,
  escapeinside = {(*}{*)},
  tabsize=3,
  morekeywords={if, and, or, is, then, else, endif, while, endwhile, for, from,
  to, do, endfor, Input, Output, Algorithmus, return},
  morecomment=[l]{//},
  columns=flexible
}

%%%%%%%%%%%%%%%%%%%%%%%%%%%%%%%%%%%%%%%%%%%%%%%%%%%%%%%%%%%%%%%%%%%%%%%%%%%%%%%
%% Style Anpassungen
%%%%%%%%%%%%%%%%%%%%%%%%%%%%%%%%%%%%%%%%%%%%%%%%%%%%%%%%%%%%%%%%%%%%%%%%%%%%%%%
%\usepackage[sc]{mathpazo}
%\renewcommand{\sfdefault}{fav}
%\setkomafont{disposition}{\sffamily}
%%%%%%%%%%%%%%%%%%%%%%%%%%%%%%%%%%%%%%%%%%%%%%%%%%%%%%%%%%%%%%%%%%%%%%%%%%%%%%%

%% Theorems %%%%%%%%%%%%%%%%%%%%%%
\usepackage[thref, hyperref, thmmarks]{ntheorem}


\theoremstyle{plain}
\theorembodyfont{\itshape}
\newtheorem{satz}{Satz}[chapter]
\newtheorem{lemma}[satz]{Lemma}
\newtheorem{kor}[satz]{Korollar}
\newtheorem{prop}[satz]{Proposition}
\newtheorem{algorithmus}[satz]{Algorithmus}
\newtheorem{folgerung}[satz]{Folgerung}


\theoremstyle{plain}
\theoremheaderfont{\normalfont\sffamily\itshape}
\theorembodyfont{\normalfont}
\newtheorem{bemerkung}[satz]{Bemerkung}
\newtheorem{beispiel}[satz]{Beispiel}
                                     
\theoremstyle{plain}
\theoremheaderfont{\normalfont\sffamily\bfseries}
\theorembodyfont{\normalfont}
\newtheorem{definition}[satz]{Definition}
\newtheorem{deflemma}[satz]{Definition/Lemma}

\newtheorem{notation}[satz]{Notation}

\theoremstyle{nonumberplain}
\theoremindent0pt
\theoremheaderfont{\sffamily\itshape}
\theorembodyfont{\normalfont}
\theoremsymbol{\ensuremath{_\square}}
\newtheorem{proof}{Beweis}
\qedsymbol{\ensuremath{_\square}}


%% Autoref Names %%%%%%%%%%%%%%%%%
\crefname{lemma}{Lemma}{Lemmas}
\crefname{equation}{Gleichung}{Gleichungen}
\crefname{definition}{Definition}{Definitionen}
\crefname{algorithmus}{Algorithmus}{Algorithmen}
\crefname{kor}{Korollar}{Korollare}
\crefname{satz}{Satz}{Sätze}
\crefname{folgerung}{Folgerung}{Folgerungen}

%% Amsmath options %%%%%%%%%%%%%%%%%
\numberwithin{equation}{chapter}
\allowdisplaybreaks

%% Pagestyle %%%%%%%%%%%%%%%%%%%%%%%%%%
\usepackage{scrpage2}
\addtokomafont{pagenumber}{\sffamily}
\addtokomafont{pagehead}{\sffamily\upshape}
\pagestyle{scrheadings}
\setcounter{secnumdepth}{5}


%% titlesec %%%%%%%%%%%%%%%%%%%%%%%%%%%%%%%%%%%%%%%%%%%%%%%%%
%\usepackage{titlesec}
  
%\titleformat{\chapter}[display]
%  {\normalfont\sffamily\huge\bfseries\color{mycol!95}}
%  {\color{gray}\chaptertitlename~\thechapter}{}{}

%\titleformat{\section}[hang]%
%  {\normalfont\sffamily\huge\bfseries\color{mycol!95}}%
%  {\color{gray}\thesection\hspace{1ex}\raisebox{-0.1\baselineskip}{\rule{1pt}{.8\baselineskip}}}{1ex}{}

%\titleformat{\subsection}[hang]%
%  {\normalfont\sffamily\Large\bfseries\color{mycol!95}}%
%  {\color{gray}\thesubsection\hspace{1ex}\raisebox{-0.1\baselineskip}{\rule{0.5pt}{.8\baselineskip}}}{1ex}{}

  
%\titleformat{\subsubsection}[hang]%
%  {\normalfont\sffamily\large\bfseries\color{mycol!95}}%
%  {\color{gray}\thesubsubsection\hspace{1ex}\raisebox{-0.1\baselineskip}{\rule{0.5pt}{.8\baselineskip}}}{1ex}{}


%% Makros %%%%%%%%%%%%%%%%%%%%%%
\let\theta\vartheta
\newcommand{\U}{\ensuremath \mathcal{U}}
\newcommand{\A}{\ensuremath \mathbb{A}}
\newcommand{\R}{\ensuremath \mathbb{R}}
\newcommand{\N}{\ensuremath \mathbb{N}}
\newcommand{\Q}{\ensuremath \mathbb{Q}}
\newcommand{\Z}{\ensuremath \mathbb{Z}}
\newcommand{\C}{\ensuremath \mathbb{C}}
\newcommand{\F}{\ensuremath \mathbb{F}}
\newcommand{\K}{\ensuremath \mathbb{K}}
\renewcommand{\P}{\ensuremath \mathbb{P}}
\newcommand{\Kb}{\ensuremath \overline K}
\renewcommand{\O}{\ensuremath \mathcal{O}}
\renewcommand{\L}{\ensuremath \mathcal{L}}
\renewcommand{\l}{\ensuremath \ell}
\renewcommand{\S}{\ensuremath\mathcal{S}}
\newcommand{\m}{\ensuremath \mathfrak{m}}
\newcommand{\speq}[1]{\ #1\ }
\newcommand{\const}{\ensuremath \mathrm{const}}
\newcommand{\divp}[1]{\ensuremath [#1]} %Divisorpunkt
\newcommand{\probn}[1]{{\sffamily #1}} %Problem Name

\let\div\undefined
\DeclareMathOperator{\charak}{char}
\DeclareMathOperator{\div}{div}
\DeclareMathOperator{\ord}{ord}
\DeclareMathOperator{\summ}{sum}
\DeclareMathOperator{\comp}{\circ}
\DeclareMathOperator{\ggT}{ggT}
\DeclareMathOperator{\kgV}{kgV}
\DeclareMathOperator{\supp}{supp}
\DeclareMathOperator{\Div0}{Div^0}
\DeclareMathOperator{\Divv}{Div}
\DeclareMathOperator{\Pic0}{Pic^0}
\DeclareMathOperator{\Gal}{Gal}
\DeclareMathOperator{\End}{End}
\DeclareMathOperator{\Ord}{Ord}
\DeclareMathOperator{\im}{im}
\DeclareMathOperator{\id}{id}
\DeclareMathOperator{\sk}{sk}


\newcommand{\funcdef}[1]{%
  \begin{array}[t]{>{\displaystyle}r>{\displaystyle}c>{\displaystyle}l}%
  #1\end{array}}



%% Others  %%%%%%%%%%%%%%%%%%%%%%%
\newcommand{\?}{{\huge \color{red} ?}}
\newcommand{\TODO}{{\sffamily\bfseries\large \color{red} TODO}}
\newcommand{\Bild}{\text{\sffamily\bfseries\large \color{red} Bild}\space}

\newcommand{\overbox}[2]{\ensuremath\begin{array}[b]{c}%
  \makebox[0pt]{\fbox{\scriptsize#2}}\\[-2pt]\text{\small$\downarrow$}\\[-3pt]%
  {\displaystyle#1}\end{array}}%

\let\marginparold\marginpar
\renewcommand{\marginpar}[1]{\marginparold{\scriptsize\sffamily #1}}

\newcommand{\inv}{^{-1}}
\let\dell\partial

\newcommand{\subk}[1]{_{(#1)}}

%% Bibliographiestil %%%%%%%%%%%%%%%%%%%%%%%%%%%%%%%%%%%%%%%%%%%%%%%%%%
\usepackage{biblatex}
\addbibresource{literatur.bib}




\begin{document}

%% Trennungen %%%%%%%%%%%%%%%%%%%%%%%%%%%%%%%%%%%%%%%%%%%%%%%%%%%%%%%%%
%%%%%%%%%%%%%%%%%%%%%%%%%%%%%%%%%%%%%%%%%%%%%%%%%%%%%%%%%%%%%%%%%%%%%%%

\frontmatter


%%%%%%%%%%%%%%%%%%%%%%%%%%%%%%%%%%%%%%%%%%%%%%%%%%%%%%%%%%%%%%%%%%%%%%%
%% Ihr Artikel                                                       %%
%%%%%%%%%%%%%%%%%%%%%%%%%%%%%%%%%%%%%%%%%%%%%%%%%%%%%%%%%%%%%%%%%%%%%%%


%% Angaben zur Standardformatierung des Titels %%%%%%%%%%%%%%%%%%%%%%%%
%\titlehead{Titelkopf }
\subject{Vorlesungszusammenfassung}
\title{Homologische Algebra}
\subtitle{gelesen von Prof. Dr. Marc Nieper-Wißkirchen}
\author{Maximilian Huber \and Stefan Hackenberg}
%\and{Der Name des Co-Autoren}
%\thanks{Fußnote}     % entspr. \footnote im Fließtext
\date{Sommersemester 2014}  % falls anderes, als das aktuelle gewünscht

%% Widmungsseite %%%%%%%%%%%%%%%%%%%%%%%%%%%%%%%%%%%%%%%%%%%%%%%%%%%%%%

\maketitle             % Titelei wird erzeugt

%% Zusammenfassung nach Titel, vor Inhaltsverzeichnis %%%%%%%%%%%%%%%%%
%\begin{abstract}
  %Was soll man sagen? Es ist Homolgische Algebra\dots
%\end{abstract}


\cleardoubleemptypage
\tableofcontents

%% Der Text %%%%%%%%%%%%%%%%%%%%%%%%%%%%%%%%%%%%%%%%%%%%%%%%%%%%%%%%%%%

%\include{intro.tex}
\mainmatter
\chapter{Simpliziale Mengen}
\section{Triangulierte Räume}

\begin{definition}
  Ein \emph{Triangulierter Raum} besteht aus
  \begin{itemize}
    \item Punkte,
    \item Kanten,
    \item Dreiecke,
    \item Tetraeder,
    \item \ldots
  \end{itemize}
\end{definition}



\begin{definition}[Inneres, induzierte Abbildung]
  Das \emph{Innere von $\Delta_n$} ist 
  \begin{align*}
    \mathring\Delta_n \speq{&:=} \begin{cases}
    \text{top. Inneres von } \Delta_n & n \geq 1 \\
    \Delta_0 & n =0
    \end{cases} \\
    \speq{&=} \{ (x_0,\ldots, x_n) \in \Delta^n \mid x_i > 0\}
  \end{align*}
  Ferner heißt 
  \[ \mathring\tau:\ \coprod_{n\geq 0} \Delta_n \times X_{(n)} \to |X| \]
  die \emph{durch $\tau$ induzierte Abbildung}.
\end{definition}


\subsection{Proposition}

\begin{prop}
  \label{prop:tau-bijektion}
  $\mathring\tau$ ist eine mengentheoretische Bijektion.
\end{prop}
\begin{proof}
  Für ein $(s,x) \in \Delta_n \times X_{(n)}$ sei sein \emph{Index} $k(s,x)$
  als die minimale Dimension einer Seite gegeben, die $s$ enthält.
  $R$-äquivalente Punkte haben den selben Index. Damit ist 
  $k: |X| \to \N_0$ eine wohldefinierte Abbildung.
  Es gilt $k(s,x) = k$, wenn $s \in \mathring\Delta_k$.
  
  Ist dann $p \in |X|$ mit $k(p) = k$, so gibt es (mind.) einen Repräsentanten
  $(s,x)$ mit $(s,x) \in \mathring\Delta_k \times X\subk k$.
  Damit ist gezeigt, dass $\mathring \tau$ surjektiv ist.

  Bleibt noch die Injektivität von $\mathring\tau$ zu zeigen:
  Seien $(s,x), (s',x') \in \coprod_{n\geq 0} \mathring\Delta_n \times X\subk n$ mit
  $(s,x) \overset R \sim (s',x')$. Zu zeigen ist damit $(s,x) = (s',x')$.
  Nach obiger Vorüberlegung ist $k(s,x) = k(s',x')$, d.h. 
  $x,x' \in X\subk k$.
  Wir haben 
  \begin{center}
  \begin{tikzcd}
    (s,x) \rar[mapsto]{f_1} & (s_1,x_1) &
    (s_2,x_2) \lar[mapsto,swap]{f_2} \rar[mapsto]{f_3} &
    (s_3,x_3) & \ldots\lar[mapsto] & (s',x') \lar[mapsto] 
  \end{tikzcd}
  \end{center}
  mit $(s_i,x_i) \in \Delta_{l_i} \times X\subk{l_i}$ und $l_i \geq k$.
  Aus dieser Kette können wir eine Kette kleinerer Länge konstruieren
  $f_1: [k] \to [l_1]$, $f_2:[l_2]\to[l_1]$ streng monoton mit 
  $s_1 = \Delta_{f_1}(s) = \Delta_{f_2}(s_2)$.
  Da $s\in \mathring\Delta_k$, liegt $s_1$ im Inneren der $f_1$-Seite von
  $\Delta_{l_1}$.
  Damit ist $\im f_2 \supseteq \im f_1$ und ergo $f_1 = f_2 \circ f$ für 
  (genau) ein streng monotones $f: [k]\to[l_2]$.

  Es gilt dann 
  \[ \Delta_{f_2}\Delta_f(s) = \Delta_{f_2\circ f}(s) = 
    \Delta_{f_1}(s) = s_1 = \Delta_{f_2}(s_2)\,. \]
  Da $\Delta_{f_2}$ injektiv ist, folgt
  $\Delta_f(s) = s_2$. Außerdem ist
  \[ X(f)(x_2) = X(f)\big( X(f_2)(x_1) \big) = 
    X(f_2\circ f)(x_1) = X(f_1)(x_1) = x \,.\]
  Also folgt:
  \begin{center}
    \begin{tikzcd}
      (s,x) \rar[mapsto]{f} \ar[mapsto, bend right]{rr}{f_3\circ f}
      & (s_2,x_2) \rar[mapsto]{f_3} & (s_3,x_3) 
      & (s_4,x_4) \lar[mapsto] \rar[mapsto] & \ldots
    \end{tikzcd}
  \end{center}
  Nach endlich vielen Schritten erhalten wir also 
  $(s,x) \overset f \longmapsto (s',x')$ mit $x,x' \in X\subk k$. Folglich ist
  $f: [k] \to [k]$ und damit die Identität, woraus die Injektivität folgt.
\end{proof}


\subsection{Skelett}

\begin{definition}[$k$-Skelett]
  Das \emph{$k$-Skelett} einer Triangulierung $(X\subk i, X(f))$ ist
  die Triangulierung 
  \[ (X\subk i, i\leq k; X(f))\,. \]
  Der zugehörige topologische Raum $\sk_k |X|$ ist das \emph{$k$-Skelett von $|X|$}.
\end{definition}


\begin{beispiel}
  Sei $(X\subk i, X(f))$
  \begin{tikzpicture}[baseline=10pt]
    \path[draw=blue, line width=1pt, fill=lime!15]
      (0,0) node[dotnode, red] {}--
      (1,0) node[dotnode, red] {}--
      (0,1) node[dotnode, red] {}-- cycle;
  \end{tikzpicture}
  so ist das $0$-Skelett 
  \begin{tikzpicture}[baseline=10pt] 
    \path
      (0,0) node[dotnode, red] {}--
      (1,0) node[dotnode, red] {}--
      (0,1) node[dotnode, red] {}-- cycle;
  \end{tikzpicture},
  das $1$-Skelett 
  \begin{tikzpicture}[baseline=10pt] 
    \path[draw=blue, line width=1pt]
      (0,0) node[dotnode, red] {}--
      (1,0) node[dotnode, red] {}--
      (0,1) node[dotnode, red] {}-- cycle;
  \end{tikzpicture}
  und das $2$-Skelett 
  \begin{tikzpicture}[baseline=10pt] 
    \path[draw=blue, line width=1pt, fill=lime!15]
      (0,0) node[dotnode, red] {}--
      (1,0) node[dotnode, red] {}--
      (0,1) node[dotnode, red] {}-- cycle;
  \end{tikzpicture}.
\end{beispiel}


\begin{kor}
  Es gilt:
  \begin{enumerate}
    \item $|X| = \sk_\infty |X| = \bigcup_{k\geq 0} \sk_k |X|$
    \item Die natürlichen Abbildungen $\sk_k |X| \to \sk_l|X|$ für
      $k\leq l$ sind abgeschlossene Einbettungen.
    \item $\sk_{k+1} |X|$ entsteht aus $\sk_k|X|$ durch Hinzufügen offener
      $(k+1)$-Simplizes, deren Ränder mit $\sk_k|X|$ verklebt werden.
  \end{enumerate}
\end{kor}
\begin{proof}
  Klar mit \thref{prop:tau-bijektion}.
\end{proof}


\subsection{Triangulation des Produktes zweier Simplizes}

Wir wollen eine kanonische Triangulierung $(X\subk n, X(f))$ von
$\Delta_p\times \Delta_q$ explizit angeben.

\begin{beispiel}
  Für $p = 1$ und $q=1$ können wir uns anschaulich folgende Triangulierung
  überlegen:
  \begin{center}
    \begin{tikzpicture}
      \path[line normal, filled]
        (0,0) node[dotnode] {} node[below] {$00$}
          --node[below] {$\Delta_p$}
        (2,0) node[dotnode] {} node[below] {10}
          -- 
        (2,2) node[dotnode] {} node[above] {11}
          --
        (0,2) node[dotnode] {} node[above] {01}
          -- node[left] {$\Delta_q$} 
        cycle;
      \path[line normal, dashed] (0,0) -- (2,2);
      \node at (1,1) {$\Delta_p\times\Delta_q$};
    \end{tikzpicture}
  \end{center}
\end{beispiel}


\begin{definition}[kanonische Triangulierung von $\Delta_p\times\Delta_q$]
  Die \emph{kanonische Triangulierung $(X\subk n, X(f))$ von 
  $\Delta_p\times\Delta_q$} ist gegeben durch:
  \begin{enumerate}
    \item Ein Element von $X\subk n$ (multidimensionale Diagonale)
      ist eine Menge von $(n+1)$ paarweise verschiedenen Paaren
      \[ \{(i_0,j_0), (i_1,j_1), \ldots, (i_n,j_n)\}\]
      mit $0\leq i_0\leq i_1 \leq\ldots\leq i_n \leq p$
      und $0\leq j_0 \leq j_1 \leq \ldots \leq j_n\leq q$.
    \item Für $f:[m] \to [n]$ streng monoton sei
      \[ X(f)(\{(i_0,j_0), \ldots, (i_n,j_n)\}) \speq=
        \{(i_{f(0)}, j_{f(0)}), \ldots, (i_{f(m)}, j_{f(m)})\} \,.\]
  \end{enumerate}
\end{definition}


\begin{definition}
  \[ \theta_n: \Delta_n \times X\subk n \to \Delta_p\times\Delta_q\]
  sodass $\theta_n(\_,x): \Delta_n \to \Delta_p\times\Delta_q \subseteq 
  \R^{p+q+2}$ diejenige lineare, ordnungserhaltende Abbildung ist, deren Bild
  $\tilde\Delta_n$ ist, wobei $\tilde\Delta_n\subseteq\R^{p+q+2}$ derjenige
  $n$-Simplex ist, der durch $(e_{ik},e_{jk}')$ aufgespannt wird für
  $x = \{(i_0,j_0),\ldots,(i_n,j_n)\}$.
\end{definition}

\begin{lemma}
  Sei $|X|$ die geometrische Realisierung von $(X\subk n, X(f))$. Dann
  existiert ein kommutatives Diagramm
  \begin{center}
    \begin{tikzcd}
      {} & \coprod\Delta_n\times X\subk n 
        \ar{dl}{\tau} \ar{dr}{\coprod\theta_n := \theta} & \\
      {} |X| \ar[dashed]{rr}{\exists!}[swap]{\tilde\varphi}& &
        \Delta_p \times\Delta_q
    \end{tikzcd}
  \end{center}
  wobei $\phi$ eine Bijektion ist.
\end{lemma}
\begin{proof}
  Da $\tau$ surjektiv, existiert höchstens ein $\varphi$. Sei 
  $(t,y) \overset f \longmapsto (s,x)$, d.h. $s = \Delta_f(t)$, 
  $y = X(f)(x)$.
  \[ \theta(s,x) = \theta(\Delta_f(t),x) = \theta(t, X(f)(x))
    = \theta(t,y)\,.\]
  Also existiert genau ein $\varphi$.

  Wir zeigen nun, dass $\theta$ surjektiv ist, damit ist auch $\varphi$
  surjektiv.
  Sei $\Delta_p = \{ (x_0,\ldots, x_p) \mid x_i\geq 0, \sum x_i = 1\}$
  und $\Delta_q = \{ (y_0,\ldots,y_p) \mid y_j\geq 0, \sum y_j = 1\}$
  Führe nun neue Koordinaten ein:
  \begin{align*}
    \xi_1 &= x_0,\quad \xi_2 = x_0+x_1,\quad\ldots,\quad 
      \xi_p = x_0 + \ldots + x_{p-1}\\
    \eta_1 &= y_0, \quad \eta_2 = y_0+y_1, \quad\ldots,\quad 
      \eta_q = y_0 + \ldots + y_{q-1}
  \end{align*}
  In diesen Koordinaten gilt:
  \[ e_i = (\underbrace{0,\ldots,0}_i, \underbrace{1,\ldots,1}_{p-i}),\quad
  e_j = (\underbrace{0,\ldots,0}_j, \underbrace{1,\ldots,1}_{q-j}) \]
  Sei $x = \{(i_0,j_0) ,\ldots, (i_{p+p},j_{p+q})\} \in X\subk{p+q}$ mit
  $(i_0,j_0) = (0,0)$ und $(i_{p+q},j_{p+q}) = (p,q)$.
  Das Bild $\theta(\Delta_{p+q}, x)$ besteht aus allen Paaren
  $\big( (\xi_1,\ldots,\xi_p),(\eta_1,\ldots,\eta_q) \big) \in 
  \Delta_p\times\Delta_q$ mit $0\leq \xi_i,\eta_j\leq 1$, wobei alle
  $\xi_i,\eta_j$ angeordnet sind und gilt:
  \begin{enumerate}[label=(\roman*)]
    \item Ist $i<j$, so steht $\xi_i$ vor $\xi_j$ und $\eta_i$ vor $\eta_j$
      und ist $j_{k+1} = j_k$, so steht an $(k+1)$-ter Stelle
      ein $\xi$, sonst ein $\eta$.
  \end{enumerate}
\end{proof}


\section{Simpliziale Mengen}

\begin{definition}[simpliziale Menge, $f$-Seite]
  Eine \emph{simpliziale Menge} ist eine Familie $X_\bullet = (X_n)_{n\geq
  0}$ von Mengen und von Abbildungen $X(f):X_n \to X_m$ für jede monotone
  Abbildung $f:[m]\to [n]$, so dass folgende Bedingungen erfüllt sind:
  \begin{enumerate}
    \item $X(\id) = \id$, 
    \item $X(f\circ g) = X(g)\circ X(f)$.
  \end{enumerate}
  Für jede monotone Abbildung $f:[m]\to [n]$ ist die \emph{$f$-Seite} die
  lineare Abbildung $\Delta_f: \Delta_m\to\Delta_n$ mit $e_i\mapsto e_{f(i)}$.
\end{definition}


\begin{beispiel}
  \makebox{}
  \begin{center}
  \begin{tikzpicture}
    \path[line normal] (0,0) node[below] {0}
      -- coordinate (a) (1,1) node[above] {1} ;
    \path[line normal] (3,0) node[below]{0}
      -- (4,0) node[below] {1}
      -- (3.5,1) node[above] {2}
      -- coordinate (b) cycle;

    \path[line normal, purple,
      shorten <= 10pt, shorten >= 10pt] 
      (a) edge[->,bend left] node[above] {$\Delta_f$} (b);
      
    \path[line normal, orange,
      shorten <= 10pt, shorten >= 10pt] 
      (a) edge[<-,bend right] node[below] {$\Delta_g$} (b);
  \end{tikzpicture}
  \quad\begin{minipage}[b]{0.3\textwidth}
    \[ f: \funcdef{ [1] &\to& [2] \\ 0&\mapsto & 0\\ 1&\mapsto & 2}\]
    \[ g: \funcdef{ [2] &\to& [1] \\ 0&\mapsto & 0\\ 0&\mapsto & 1}\]
    \end{minipage}
  \end{center}
\end{beispiel}

\begin{bemerkung}
  $\Delta_f$ ist im Allgemeinen keine Einbettung mehr!
\end{bemerkung}

\begin{definition}
  Die \emph{geometrische Realisierung $|X_\bullet|$} einer simplizialen Menge
  $X_\bullet$ ist der topologische Raum mit zugrundeliegender Menge
  \[ \big( \coprod \Delta_n \times X_n \big) \big/ R\,,\]
  wobei $R$ die schwächste Äquivalenzrelation ist, für die
  \[ (s,x) R (t,y) \quad\Longleftarrow\quad
    y = X(f)(x),\ s = \Delta_f(t)\]
  für alle monotonen Abbildungen $f:[m]\to [n]$ gilt. Die Topologie auf $|X|$ ist
  wieder die Quotiententopologie.
\end{definition}

\subsection{Beispiele}
\subsubsection{Nerv einer Überdeckung}

Sei $(\U_\alpha)_{\alpha \in A}$ eine Überdeckung eines topologischen Raumes
$Y$ durch offene (bzw. abgeschlossene Teilmengen). Sei
\[ X_n := \{ (\alpha_0, \ldots, \alpha_n) \in A^{n+1} \mid 
  \U_{\alpha_0} \cap \ldots \cap \U_{\alpha_n} \neq \emptyset \}.\]

\begin{definition}
  $X_\bullet$ heißt \emph{Nerv von $(\U_\alpha)_{\alpha\in A}$}.
\end{definition}

\begin{beispiel}
  \Bild Überdeckung der $S^1$.
  \ldots mit der geometrischen Realisierung:
  \begin{center}
    \begin{tikzpicture}[scale=2]
      \path (0,0) node[dotnode] (a) {} node[left]{(2), (2,2)}
      -- (2,0) node[dotnode] (b) {} node[right]{(1), (1,1)}
      -- (1,1) node[dotnode] (c) {} node[above]{(0), (0,0)}; 
      \path[line normal, fill=purple!10] 
        (a) to[bend left=20] node[below=-3pt] {(1,2)} (b)
          to[bend left=20] node[below] {(2,1)} (a);
      \path[line normal, fill=purple!10] 
        (b) to[bend left=20] node[sloped,above=-3pt] {(0,1)} (c)
          to[bend left=20] node[sloped,above] {(1,0)} (b);
      \path[line normal, fill=purple!10] 
        (a) to[bend left=20] node[sloped, above] {(0,2)} (c)
          to[bend left=20] node[sloped, above=-3pt] {(2,0)} (a);
    \end{tikzpicture}
  \end{center}
\end{beispiel}

\begin{bemerkung}
  Ist die Überdeckung lokal endlich und sind die nicht-leeren Durchschnitte
  zusammenziehbar, so ist $|X|$ homotopieäquivalent zu $Y$.
\end{bemerkung}


\subsubsection{Singuläre Simplizes}

\begin{definition}
  Sei $Y$ ein topologischer Raum. Ein \emph{singulärer $n$-Simplex von $Y$}
  ist eine stetige Abbildung $\varphi:\Delta_n \to Y$.
  \[ X_n := \{\varphi:\Delta_n \to Y\text{ sing. $n$-Simplizes}\}\]
  und
  \[ X(f)(\varphi) := \varphi\circ \Delta_f\]
  für alle $f:[m] \to [n]$ monoton. Dies definiert eine simpliziale Menge
  $X_\bullet$.
\end{definition}

\begin{bemerkung}
  $X_\bullet$ ist riesig!
\end{bemerkung}

\begin{bemerkung}
  In einem gewissen Sinn sind \emph{singuläre simpliziale Menge eines
  topologischen Raums bilden} und \emph{geometrische Realisierung einer
  simplizialen Menge bilden} zusammengehörige Prozesse; sie lösen jeweils ein
  Optimierungsproblem, das der andere Prozess stellt. Das Schlagwort dazu ist
  die allgemeine \emph{Adjunktion zwischen Nerv und Realisierung}, und
  vielleicht werden wir dazu später mehr erfahren.
\end{bemerkung}

\subsubsection{Die simpliziale Menge $\Delta[p]$}

\begin{definition}
  Sei
  \begin{align*}
    \Delta[p]_n \speq{&:=} \{ g: [n]\to [p] \text{ monoton}\}\,, \\
    \Delta[p](f)(g) \speq{&:=} g\circ f.
  \end{align*}
  $\Delta[p]_\bullet$ heißt \emph{simplizialer $p$-Simplex}.
\end{definition}

\begin{lemma}
  Es existiert ein kanonischer Homöomorphismus $\Delta_p \to |\Delta[p]|$.
\end{lemma}
\begin{proof}
  Übungsaufgabe.
\end{proof}


\subsubsection{Die einem Verklebedatum zugeordnete simpliziale Menge}

Sei $(X\subk n, X(f))$ ein Verklebedatum. Dazu gehört die simpliziale Menge
$\tilde X_\bullet$ mit
\begin{align*}
  \tilde X_n \speq{&:=} \{ (x,g) \mid x\in X\subk k,\ g:[m]\to [k] 
    \text{ monoton, surjektiv}\}\\
  \tilde X(f) \speq{&:=} \tilde X_m \to \tilde X_n,\ 
    (x,g) \mapsto (X(f_1)(x), f_2)
\end{align*}
für $f: [n]\to [m]$ monoton, mit $g\circ f = f_1 \circ f_2$ für
$f_1,f_2$ monoton, $f_1$ injektiv, $f_2$ surjektiv.

\begin{lemma}
  $\tilde X_\bullet$ ist in der Tat eine simpliziale Menge.
\end{lemma}
\begin{proof}
  Übungsaufgabe.
\end{proof}

\begin{bemerkung}
  Später werden wir sehen, dass
  \[ |\tilde X_\bullet| \speq= |X| \,.\]
\end{bemerkung}

\begin{prop}
  Eine simpliziale Menge $\tilde X$ kann genau dann aus einem Verklebedatum $X$
  erhalten werden, wenn für jeden nicht-degenerierten Simplex $x \in \tilde
  X_n$ und für jede streng monotone Abbildung $f:[m]\to [n]$ der Simplex
  $\tilde X(f)(x)$ ebenfalls nicht-degeneriert ist. In diesem Fall ist $X$ im
  wesentlichen durch $\tilde X$ bestimmt.
\end{prop}
\begin{proof}
  Übungsaufgabe.
\end{proof}

\subsubsection{Der klassifizierende Raum einer Gruppe}
Sei $G$ eine Gruppe. Setze $(BG)_n := G^n$ und für $f:[m]\to[n]$ monoton,
setzen wir $BG(f): G^n\to G^m,\ (g_1,\ldots,g_n)\mapsto (h_1,\ldots,h_m)$,
wobei
\[ h_i := \prod_{j = f(i-1)+1}^{f(i)} g_j\,.\]


\begin{beispiel}
  $f: [3]\to[4], 0\mapsto 0, 1,2\mapsto 2, 3\mapsto 4$ 
  \Bild
\end{beispiel}

\begin{definition}
  Die geometrische Realisierung $|BG|$ heißt der \emph{klassifizierende Raum
  von $G$}.
\end{definition}


\subsubsection{Nicht-ausgeartete Simplizes}
\label{subsub:nicht_ausgeartet}
\begin{definition}[ausgearteter Simplex]
  Sei $X$ eine simpliziale Menge. Ein $n$-Simplex $x \in X_n$ heißt 
  \emph{ausgeartet}, falls $x = X(f)(y)$ für $f:[n]\to[m]$ mit $m<n$,
  $y\in X_m$ und $f$ surjektiv.
\end{definition}

\begin{bem}
  Ist $x$ nicht-ausgeartet und $x = X(f)(y)$ für ein $f$ und $y$, so ist
  $f$ eine Injektion. Ansonsten zerlege $f = f_1\circ f_2$ mit $f_1$ injektiv
  und $f_2$ surjektiv. Damit $x = X(f_1\circ f_2)(y) = X(f_2)(X(f_1)(y))$.
\end{bem}

Ist $X$ eine simpliziale Menge, so sei 
$X\subk n := \{ x\in X_n \mid x\text{ nicht-ausgeartet}\}$ und
\begin{center}
  \begin{tikzcd}
    \tau: \coprod_{n\geq 0} \Delta_n \times X_n \rar & {} |X| \\
    \coprod_{n\geq 0} \mathring\Delta_n \times X\subk n
    \uar[hookrightarrow] \urar{\mathring\tau}
  \end{tikzcd}
\end{center}

\begin{prop}
  \label{prop:tauo_bijektion} 
  $\mathring\tau$ ist eine Bijektion.
\end{prop}

Um diese Proposition zu beweisen, brauchen wir einige Lemmas.

\begin{lemma}
  \label{lemma:ex_eind_paar}
  Für jedes $x \in X_n$ existiert ein eindeutiges Paar $(f,y)$, mit
  $y \in X\subk m$, $f:[n]\to[m]$ monoton, surjektiv, so dass $x = X(f)(y)$.
\end{lemma}
\begin{proof}
  Die Existenz folgt aus der Definition von $X\subk m$.
  Zur Eindeutigkeit: Seien $(f,y)$, $f:[n]\to[m]$ und $(f',y')$,
  $f':[n]\to [m']$ zwei solcher Paare mit $x = X(f)(y) = X(f')(y')$.
  Sei $g:[m]\to[n]$ ein monotoner Schnitt von $f$, d.h. 
  $f\circ g = \id_{[m]}$. Dann ist 
  \[ y = X(\id)(y) = X(f\circ g)(y) = X(g)X(f)(y) = X(g)(x). \]
  Damit $y = X(g)X(f')(y') = X(f'\circ g)(y')$.
  Da $y$ nicht-entartet ist, ist $f'\circ g:[m]\to [m']$ eine Injektion, so
  dass $m\leq m'$. Analog $m'\leq m$, also $m = m'$. Damit ist
  $f'\circ g$ eine monotone Injektion von $[m]$ nach $[m]$, also
  $f'\circ g = \id_{[m]}$, also $y = y'$. Da $f'\circ g = \id$ für alle
  Schnitte $g$ von $f$, folgt $f' = f$.
\end{proof}


\begin{folgerung}
  \label{folgerung:aufteilen_in_surj_inj}
  Sei $x \in X_n$ ein $n$-Simplex, $y\in X_m$ nicht-entartet, 
  $f:[n]\to[m]$ eine monotone Surjektion mit $x = X(f)(y)$. Sei weiter 
  $z \in X_l$ und $g: [n]\to[l]$ eine monotone Surjektion mit $x=X(g)(z)$. Dann
  faktorisiert $f$ als $f = h\circ g$ mit einem $h:[l]\to[m]$, so dass
  $z = X(h)(y)$.
\end{folgerung}
\begin{proof}
  Sei $(h',y')$ das Paar aus \cref{lemma:ex_eind_paar} mit $z = X(h')(y')$.
  Dann ist $x = X(g)(z) = X(g)(X(h')(y')) = X(h'\circ g)(y')$. Damit ist
  $h'\circ g = f$ und $y = y'$. Setze $h:= h'$.
\end{proof}

\paragraph{Beweis von \cref{prop:tauo_bijektion}}

\begin{proof}[von \cref{prop:tauo_bijektion}]
  \begin{enumerate}
    \item $\mathring\tau$ ist surjektiv: Sei $p\in |X|$. Sei
      $k$ die kleinste Dimension, so dass ein $(s,x) \in\Delta_k\times X_k$ mit
      $\tau(s,x) = p$ existiert. Wir wollen nun zeigen, dass
      $(s,x) \in \mathring\Delta_k \times X\subk k$: Ist $k=0$, so ist 
      $\Delta_0 = \mathring\Delta_0$ und $X\subk 0 = X_0$, also nichts zu
      zeigen. Ist $x$ uasgeartet, so ist $x = X(f)(y)$, $f:[k]\to[l]$ monoton,
      $l < k$. Dann ist $\tau(s,x) = \tau(\Delta_f(s),y)$. Widerspruch zur
      Minimalität von $k$. Also $x \in X\subk k$. Ist
      $s\notin\mathring\Delta_k$, so existiert ein injektives $f:[l]\to[k]$
      monoton mit $s = \Delta_f(t)$ für $t\in \Delta_l$. Damit 
      $\tau(s,x) = \tau(\Delta_f(t),x) = \tau(t,X(f)(x))$ im Widerspruch zu 
      $l<k$.
    \item $\mathring\tau$ ist injektiv: Seien 
      $(s,x) \in \mathring\Delta_k \times X\subk k$,
      $(s',x')\in\mathring\Delta_l\times X\subk l$ mit $\tau(s,x) =
      \tau(s',x') (\ast)$, so ist zu zeigen, dass $s = s'$ und $x = x'$.
      Wegen $(\ast)$ existiert 
      \[ (s,x) = (s_0,x_0) \sim (s_1,x_1) \sim \ldots\sim (s_N,x_N) = (s',x')\]
      wobei $\sim$ für $(s_i,x_i) \overset{f_i^+}{\mapsto} (s_{i+1},x_{i+1})$
      oder $(s_i,x_i) \overset{f_i^-}{\leftmapsto} (s_{i+1},x_{i+1})$ 
      steht. Ohne Einschränkung können wir ``Zickzack'' annehmen.
      Wir zeigen: Ist $N=1$, so ist $f_0 = \id$ und ist $N>1$, so existiert
      eine kürzere Kette.
    \item Beh: Da $(s,x)\in\mathring\Delta_k \times X\subk k$, so ist
      $f_0^+$ eine Injektion und $f_0^-$ eine Surjektion.
      Im Fall $f_0^+$ ist $x = X(f_0^+)(x_1)$. Dann ist $f_0^+$ eine Injektion
      nach \autoref{subsub:nicht_ausgeartet}. Im Fall $f_0^-$ ist
      $s = \Delta_{f_0^-}(s_1)$. Da $s \in \mathring\Delta_k$, muss $f_0^-$
      surjektiv sein. Wir folgern: $N=1$.\autoref{subsub:nicht_ausgeartet}. Im
      Fall $f_0^-$ sit
      $s = \Delta_{f_0^-}(s_1)$. Da $s \in \mathring\Delta_k$, muss $f_0^-$
      surjektiv sein. Wir folgern: Für $N=1$ ist $f_0^- / f_0^+$ ist injektiv
      und surjektiv, also gleich $\id$.
    \item Für $N\geq 2$ betrachte
      \begin{center}
        \begin{tikzcd}
          (s_i,x_i) \rar[mapsto]{f_i^+} & (s_{i+1},x_{i+1}) &
            (s_{i+2},x_{i+2}) \lar[mapsto]{f_{i+1}^-}\\[5pt]
          {}[m_i] \rar &{} [m_{i+1}] &{} [m_{i+2}] \lar
        \end{tikzcd}
      \end{center}
      Sei außerdem $f_i^+$ injektiv. Wir zeigen, dass dann folgende Kette
      existiert:
      \begin{center}
        \begin{tikzcd}
          (s_i,x_i) & (s_{i+1},x_{i+1}) \lar[mapsto]{g} \rar[mapsto]{h}&
            (s_{i+2},x_{i+2})\\
          {}[m_i] \rar &{} [m_{i+1}] &{} [m_{i+2}] \lar
        \end{tikzcd}
      \end{center}
      Sei $ I:=(f_{i+1}^-)\inv(f_i^+([m_i]))$, $l:= |I| -1$. Sei 
      $h:[l]\to[m_{i+2}]$ die injektive monotone Abbildung mit Bild $I$
      Dann existiert genau ein $g:[l]\to[m_i]$ mit $f_i^+\circ g = 
      f_{i+1}^- \circ h$. Da $\im h=I$, ist 
      $\im\Delta_h = (\Delta_{f_{i+1}^-)\inv(\Delta_{f_i^-}(\Delta_{m_i}))$.
      Da $\Delta_{f_{i+1}^-}(s_{i+2}) = s_{i+1} = \Delta_{f_i^+}(s_i)$,
      existiert ein $t\in \Delta_l$ mit $\Delta_h(t) = s_{i+2}$. Außerdem:
      \[ \Delta_{f_i^+}(\Delta_g(t)) = \Delta_{f_{i+1}^-}(\Delta_h(t))
        = s_{i+1} = \Delta_{f_i^+}(s_i)\]
      Da $\Delta_{f_i^+}$ injektiv, folgt $\Delta_g(t) = s_i$. Setze
      $y := X(h)(x_{i+2}) = X(g)(x_i)$.
    \item Sei $N\geq 2$, so beginnt die Kette mit 
      $(s,x) \overset{f_0^+}{\mapsto} (s_1,x_1) \overset{f_1^-}\leftmapsto
      (s_2,x_2)$, so ist $f_0^+$ nach (4) eine Injektion. Nach (3) erhalten wir
      dann eine Kette der Länge 2 (für $N=2$ oder $N-1$ für $N\geq 3$) der Form
      \[ (s_0,x_0) \overset{f_0^-}{\leftmapsto} (s_1,x_1) 
        \overset{f_1^+}{\mapsto} (s_2,x_2)\]
      Nach (4) ist $f_0^-$ eine Surjektion.
    \item Ist $N = 2$, also $(s_2,x_2) \in \mathring\Delta_l\times X\subk l$,
      so ist $f_1^+$ surjektiv nach (3).
      Da $x_0,x_2$ nicht-ausgeartet, folgt nach \cref{lemma:ex_eind_paar}, dass
      $f_0^- = f_1^+$ und $x_0 = x_2$, also $s_0 = s_2$.
    \item Sei also $N\geq 3$. Schreibe $f_1^+ = i\circ p$, wobei 
      $p:[m_1] \to[l]$ monotone Surjektion und $i:[l]\to[m_2]$ monotone
      Injektion. Da $x_0$ nicht-ausgeartet und $f_0^-$ surjektiv, existiert
      nach \cref{folgerung:aufteilen_in_surj_inj} 
      eine Faktorisierung $f_0^- = g \circ p$ für ein 
      $g:[l]\to[k]$. Damit ersetzen wir 
      $(s_0,x_0)\overset{f_0^-}{\leftmapsto} (s_1,x_1) \overset{f_1^+}{\mapsto}
      (s_2,x_2)$ durch
      \[ (s_0,x_0) \overset{g}{\leftmapsto} (\Delta_p(s_1, X(g)(x_0))
        \overset{i}{\mapsto} (s_2,x_2)\,,\]
      d.h. ohne Einschränkung ist $f_1^+$ eine Injektion. Nach (4) können wir
      also $(s_1,x_1) \overset{f_1^+}{\mapsto} (s_2,x_2) 
      \overset{f_2^-}{\leftmapsto} (s_3,x_3)$ durch
      $(s_1,x_1) \overset{f_1^-}{\leftmapsto} (s_2,x_2)
      \overset{f_2^+}{\mapsto} (s_3,x_3)$ ersetzen und wir erhalten eine Kette
      der Länge $N-2$.
  \end{enumerate}
\end{proof}

\begin{folgerung}
  Sei $|X|$ ein triangulierter Raum mit Verklebedatum $(X\subk n, X(f))$. Sei
  $\tilde X$ die dazugehörige simpliziale Menge Dann $|\tilde X| = |X|$.
\end{folgerung}
\begin{proof}
  \makebox{}
  \begin{center}
    \begin{tikzcd}
      \coprod_{n\geq 0} \Delta_n\times \tilde X_n \dar
        \rar{(s,\tilde x) = (s,g) \mapsto (\Delta_g(s),x)}
        &[4cm] \coprod_{n\geq 0} \Delta_n \times X\subk n \dar
        \\ {}
      |\tilde X| \rar[dashed]{\varphi}
        &{} |X|
    \end{tikzcd}
  \end{center}
  Dann ist nach Übungsaufgabe $\varphi$ ein Homöomorphismus.
\end{proof}

%\subsection{Skelett und Dimension}
\subsection{Abbildungen simplizialer Mengen}
\subsection{Verfeinerung von Überdeckungen}
\subsection{Stetige Abbildungen}
\subsection{Gruppenhomomorphismen}
\section{Simpliziale Topologische Räume und Eilenberg-Zilber}
\subsection{Drei Beschreibungen von $\Delta_p\times\Delta_q$}
\subsection{Definition}
\subsection{Definition}
\subsection{Beispiel}
\subsection{Definition}
\subsection{Die geometrische Realisierung einer bisimplizialen Menge}
\subsection{Satz von Eilenberg und Zilber}
\subsection{Beweisidee}


%\subsection{Aufeinanderfolgende Faktorisierungen}
\marginpar{Tiefe der Nummerierung unklar!}

$Z$ sei beliebige Menge, $R_1,R_2$ zwei Äquivalenzrelationen auf $Z$. Sei
$R$ die von $R_1,R_2$ erzeugte Äquivalenzrelationen.
Sei $R_1/R_2$ folgende Äquivalenzrelationen auf $Z/R_1$:
\[ (x \bmod R_1) \overset{R_2/R_1}{\sim} (y \bmod R_1) 
  \speq{:\Leftrightarrow} \exists x',y'\in Z:\ 
  x'\sim y' \bmod R_2,\ x'\sim x\bmod R_1\ y'\sim y\bmod R_1\,.\]
Dann sind
\begin{enumerate}[label=\alph*)]
  \item $(Z/R_1)\big/(R_2/R_1)$,
  \item $(Z/R_2)\big/ (R_1/R_2)$ und 
  \item $Z/R$
\end{enumerate}
kanonisch bijektiv und falls a),b),c) die Quotiententopologie bzgl. einer
Topologie auf $Z$ tragen, sind diese Bijektionen Homöomorphismen.

\begin{lemma}
  Sei $R$ die von $R\I$, $R\II$ erzeugte Äquivalenzrelationen. Dann ist
  \[ Z\big/ R \speq\cong \coprod_{k\geq 0} (X_{kk} \times \Delta_k)
  \quad\text{und}\quad (Z/R)\big/(R\D/R) \cong |X|\D\]
\end{lemma}
\begin{proof}
  Sei $\tilde p: Z \to \coprod_{k\geq 0}(X_{kk}\times \Delta_k)$, 
  $(x,(f,g),s) \mapsto (X(f,g)(x),s)$.
  $R\I$ und $R\II$-äquivalente Punkte haben das selbe Bild unter $\tilde
  p$. Damit erhalten wir ein wohldefiniertes $p: Z/R \to \coprod_{k\geq 0}
  (X_{kk}\times \Delta_k)$. Betrachte
  \[\tilde i:\coprod_{k\geq 0} (X_{kk}\times \Delta_k) \to Z,\ 
    (x,s)\mapsto (x, (\id,\id),s)\,. \]
  und $i = (Z\to Z/R) \circ i$.
  Dann ist $p\circ i = \id$. Damit ist $p$ surjektiv. 
  Jeder Punkt $(x,(f,g),s)$ ist $R$-äquivalent zu $(X(f,g)x, (\id,\id),s)$.
  Damit ist $p$ auch injektiv, also ein Homöomorphismus.
  
  Berechnung von $R\D/R$ auf $Z/R$:
  \begin{center}
    \begin{tikzcd}
      (x,(f,g),s) \rar[mapsto]{p} \dar[leftrightarrow]{\sim_h}
        & (X(f,g)(x),s) \dar[leftrightarrow]{\sim_{R\D/R}} 
        \rar[equal] &[-25pt]  
        (X(h,h) \circ X(f',g')(x),s)  \\
      (X, (f',g'),s') \rar[mapsto]{p} &
        (X(f',g')(x),s') \rar[equal] & (X(f',g')(x), \Delta_h(s))
    \end{tikzcd}
  \end{center}
  D.h. $R\D/R: (X(h,h)(x),s) \sim (x,\Delta_h(s))$ und es folgt
  $\coprod_{k\geq 0} X_{kk} \times \Delta_k = |X|\D$.
\end{proof}


\begin{lemma}
  \label{lemma:11}
  Die geometrische Relaisierung $|D[m,n]|$ ist kanonisch homöomorph zu
  $\Delta_m\times \Delta_n$.
\end{lemma}
\begin{proof}
  Einem $k$-Simplex in $D[m,n]$, d.h. einem Paar $(f,g)$ monotoner
  Abbildungen $f:[k]\to[m]$, $g:[k]\to [n]$, ordnen wir das $(k+1)$-Tupel
  $((i_0,j_0),\ldots,(i_k,j_k))$ mit $i_l = f(l)$, $j_l=g(l)$ für 
  $0\leq l\leq k$ zu. Der Simplex $(f,g)$ ist genau dann nicht-degeneriert,
  wenn alle Paare verschieden sind. Außerdem: Ist $h:[k']\to[k]$ injektiv und
  $(f,g)$ nicht ausgeartet, so ist $D[m,n](h)(f,g) = (f\circ h, g\circ h)$
  ebenfalls nicht ausgeartet. Damit ist
  \[|D[m,n]| = |\text{Verklebedatum der nicht-ausgearten Simplizes}|
    \cong \Delta_m\times\Delta_n\]
\end{proof}


\begin{folgerung}
  Es gilt: 
  \[ Z\big/ R\D \speq\cong \coprod_{m,n} X_{m,n}\times \Delta_m 
    \times \Delta_n\]
  und
  \[ R\I \big/R\D: (x,s,t) \sim(x',s',t),
    \text{ falls } s'=\Delta_h(s),\ x = X(h,\id)(x')\,,\]
  für ein monotones $h$. $R\II/R\D$ analog.
\end{folgerung}
\begin{proof}
  $Z/R\D \cong\coprod_{m,n\geq 0} X_{m,n} \times (\Delta_m\times \Delta_n)$
  ist \thref{lemma:11}.
  \[ F_h: D[m,n]_h \to D[m',n]_h,\ (f,g) \mapsto (h\circ f, g)\]
  so ist 
  \begin{center}
  \begin{tikzcd}
    {|F_h|}: &[-20pt] {|D[m,n]|} \dar{\cong} \rar & {|D[m',n]|} \dar{\cong}
    \\
    & \Delta_m \times\Delta_n \rar{\Delta_n\times\Delta_\id} &
      \Delta_{m'} \times\Delta_n
  \end{tikzcd}
  \end{center}
\end{proof}


\begin{notation}
  \begin{itemize}
    \item $\tilde Z := Z/R\D$
    \item $\tilde R\I := R\I / R\D$
    \item $\tilde R\II := R\II / R\D$
\end{notation}

\begin{folgerung}
  Es gilt
  \[ (\tilde Z / \tilde R\I)\big/ (\tilde R\II/\tilde R\I) 
    \speq= |X|\I \quad\text{und}\quad
    (\tilde Z/\tilde R\II) \big/ (\tilde R\I / \tilde R\II) 
    \speq= |X|\II.\]
\end{folgerung}


\section{Homologie und Kohomologie}

\subsection{Ketten und Koketten}

\begin{definition}
  Eine \emph{$n$-dimensionale Kette} (\emph{$n$-Kette}) einer simplizialen
  Menge $X$ ist ein Element der freien abelschen Gruppe $C_n(X)$, welche von
  allen $n$-Simplizes erzeugt wird.
\end{definition}


\begin{definition}
  Der \emph{Rand einer $n$-Kette} $c\in C_n(X)$ ist die $(n-1)$-Kette
  $\d_n c$, definiert durch
  \[ \d_n\left( \sum_{x\in X_n} a(x) x \right) \speq{:=}
    \sum_{x\in X_n} a(x) \sum_{i=0}^n (-1)^i X(\dell_n^i)(x)\]
  Damit ist der \emph{Randoperator} $\d_n:C_n(X)\to C_{n-1}(X)$ ein
  Gruppenhomomorphismus. Für $n=0$ setze $C_{-1}(X) = 0$, $\d_0 = 0$.
\end{definition}

\begin{bemerkung}
  Anstelle von $\Z$ hätten wir auch Koeffizienten in einer beliebigen
  abelschen Gruppe $A$ verwenden können:
  $C_n(X,A) := C_n(X)\otimes_\Z A$, insb. $C_n(X) = C_n(X,\Z)$.
  und $\d_n. C_n(X,A) \to C_{n-1}(X,A)$ ist ein $A$-Modulhomomorphismus.
\end{bemerkung}


\begin{definition}
  Eine \emph{$n$-Kokette (mit Koeffizienten in $A$)} ist ein Element in
  \[ C^n(X,A) \speq{:=} \{ f: X_n \to A\ \text{Abbildung}\}\]
  Der \emph{Korandoperator} $\d^n: C^n(X,A) \to C^{n+1}(X,A)$ ist definiert
  durch 
  \[ \d^n(f)(x) \speq{:=} \sum_{i=0}^{n+1} (-1)^i f(X(\dell_{n+1}^i(x)))\]
  für $x \in X_{n+1}$.
\end{definition}


\begin{bemerkung}
  Man kann $C_n(X,A) \hookrightarrow C^n(X,A)$ sehen, was aber nur unter
  Vorsicht zu genießen ist. Sie sind nämlich bezüglich Rand- und Korandoperator
  nicht verträglich!
\end{bemerkung}


\begin{lemmea}
  \begin{enumerate}[label=(\alph*)]
    \item $\d_{n-1}\circ \d_n= 0$.
    \item $\d^{n+1}\circ \d^n=0$.
  \end{enumerate}
\end{lemmea}
\begin{proof}
  Für $0\leq j< i\leq n-1$ ist 
  $\dell_n^i\circ \dell_{n-1}^j = \dell_n^j\circ\dell_{n-1}^{i-1}:
  [n-2]\to[n]$ mit $i,j\notin \im$.
  \begin{align*}
    \d_{n-1}(\d_n(x)) &= \sum_{j=0}^{n-1}\sum_{i=0}^n (-1)^{i+j}
     X(\dell_{n-1}^j)X(\dell_n^i)(x) \\
    &= \sum_{j=0}^{n-1}\sum_{i=0}^n (-1)^{i+j} X(\dell_n^i\circ
    \dell_{n-1}^j)(x) = 0
  \end{align*}
\end{proof}

%\subsection{Komplexe}
\subsection{Definition}
\subsection{Die Geometrie von Ketten}
\subsection{Die Geometrie von Koketten}
\subsection{Koeffizientensysteme}

%\input{vorlesung_08}
%\input{vorlesung_09}
%\input{vorlesung_10}
%\input{vorlesung_11}
% 



\nocite{*}
\printbibliography
\end{document}
