\subsubsection{Aufeinanderfolgende Faktorisierungen}
\marginpar{Tiefe der Nummerierung unklar!}

$Z$ sei beliebige Menge, $R_1,R_2$ zwei Äquivalenzrelationen auf $Z$. Sei
$R$ die von $R_1,R_2$ erzeugte Äquivalenzrelationen.
Sei $R_1/R_2$ folgende Äquivalenzrelationen auf $Z/R_1$:
\[ (x \bmod R_1) \overset{R_2/R_1}{\sim} (y \bmod R_1) 
  \speq{:\Leftrightarrow} \exists x',y'\in Z:\ 
  x'\sim y' \bmod R_2,\ x'\sim x\bmod R_1\ y'\sim y\bmod R_1\,.\]
Dann sind
\begin{enumerate}[label=\alph*)]
  \item $(Z/R_1)\big/(R_2/R_1)$,
  \item $(Z/R_2)\big/ (R_1/R_2)$ und 
  \item $Z/R$
\end{enumerate}
kanonisch bijektiv und falls a),b),c) die Quotiententopologie bzgl. einer
Topologie auf $Z$ tragen, sind diese Bijektionen Homöomorphismen.

\begin{lemma}
  Sei $R$ die von $R\I$, $R\II$ erzeugte Äquivalenzrelationen. Dann ist
  \[ Z\big/ R \speq\cong \coprod_{k\geq 0} (X_{kk} \times \Delta_k)
  \quad\text{und}\quad (Z/R)\big/(R\D/R) \cong |X|\D\]
\end{lemma}
\begin{proof}
  Sei $\tilde p: Z \to \coprod_{k\geq 0}(X_{kk}\times \Delta_k)$, 
  $(x,(f,g),s) \mapsto (X(f,g)(x),s)$.
  $R\I$ und $R\II$-äquivalente Punkte haben das selbe Bild unter $\tilde
  p$. Damit erhalten wir ein wohldefiniertes $p: Z/R \to \coprod_{k\geq 0}
  (X_{kk}\times \Delta_k)$. Betrachte
  \[\tilde i:\coprod_{k\geq 0} (X_{kk}\times \Delta_k) \to Z,\ 
    (x,s)\mapsto (x, (\id,\id),s)\,. \]
  und $i = (Z\to Z/R) \circ i$.
  Dann ist $p\circ i = \id$. Damit ist $p$ surjektiv. 
  Jeder Punkt $(x,(f,g),s)$ ist $R$-äquivalent zu $(X(f,g)x, (\id,\id),s)$.
  Damit ist $p$ auch injektiv, also ein Homöomorphismus.
  
  Berechnung von $R\D/R$ auf $Z/R$:
  \begin{center}
    \begin{tikzcd}
      (x,(f,g),s) \rar[mapsto]{p} \dar[leftrightarrow]{\sim_h}
        & (X(f,g)(x),s) \dar[leftrightarrow]{\sim_{R\D/R}} 
        \rar[equal] &[-25pt]  
        (X(h,h) \circ X(f',g')(x),s)  \\
      (X, (f',g'),s') \rar[mapsto]{p} &
        (X(f',g')(x),s') \rar[equal] & (X(f',g')(x), \Delta_h(s))
    \end{tikzcd}
  \end{center}
  D.h. $R\D/R: (X(h,h)(x),s) \sim (x,\Delta_h(s))$ und es folgt
  $\coprod_{k\geq 0} X_{kk} \times \Delta_k = |X|\D$.
\end{proof}


\begin{lemma}
  \label{lemma:11}
  Die geometrische Relaisierung $|D[m,n]|$ ist kanonisch homöomorph zu
  $\Delta_m\times \Delta_n$.
\end{lemma}
\begin{proof}
  Einem $k$-Simplex in $D[m,n]$, d.h. einem Paar $(f,g)$ monotoner
  Abbildungen $f:[k]\to[m]$, $g:[k]\to [n]$, ordnen wir das $(k+1)$-Tupel
  $((i_0,j_0),\ldots,(i_k,j_k))$ mit $i_l = f(l)$, $j_l=g(l)$ für 
  $0\leq l\leq k$ zu. Der Simplex $(f,g)$ ist genau dann nicht-degeneriert,
  wenn alle Paare verschieden sind. Außerdem: Ist $h:[k']\to[k]$ injektiv und
  $(f,g)$ nicht ausgeartet, so ist $D[m,n](h)(f,g) = (f\circ h, g\circ h)$
  ebenfalls nicht ausgeartet. Damit ist
  \[|D[m,n]| = |\text{Verklebedatum der nicht-ausgearten Simplizes}|
    \cong \Delta_m\times\Delta_n\]
\end{proof}


\begin{folgerung}
  Es gilt: 
  \[ Z\big/ R\D \speq\cong \coprod_{m,n} X_{m,n}\times \Delta_m 
    \times \Delta_n\]
  und
  \[ R\I \big/R\D: (x,s,t) \sim(x',s',t),
    \text{ falls } s'=\Delta_h(s),\ x = X(h,\id)(x')\,,\]
  für ein monotones $h$. $R\II/R\D$ analog.
\end{folgerung}
\begin{proof}
  $Z/R\D \cong\coprod_{m,n\geq 0} X_{m,n} \times (\Delta_m\times \Delta_n)$
  ist \thref{lemma:11}.
  \[ F_h: D[m,n]_h \to D[m',n]_h,\ (f,g) \mapsto (h\circ f, g)\]
  so ist 
  \begin{center}
  \begin{tikzcd}
    {|F_h|}: &[-20pt] {|D[m,n]|} \dar{\cong} \rar & {|D[m',n]|} \dar{\cong}
    \\
    & \Delta_m \times\Delta_n \rar{\Delta_n\times\Delta_\id} &
      \Delta_{m'} \times\Delta_n
  \end{tikzcd}
  \end{center}
\end{proof}


\begin{notation}
  \begin{itemize}
    \item $\tilde Z := Z/R\D$
    \item $\tilde R\I := R\I / R\D$
    \item $\tilde R\II := R\II / R\D$
\end{notation}

\begin{folgerung}
  Es gilt
  \[ (\tilde Z / \tilde R\I)\big/ (\tilde R\II/\tilde R\I) 
    \speq= |X|\I \quad\text{und}\quad
    (\tilde Z/\tilde R\II) \big/ (\tilde R\I / \tilde R\II) 
    \speq= |X|\II.\]
\end{folgerung}


\section{Homologie und Kohomologie}

\subsection{Ketten und Koketten}

\begin{definition}
  Eine \emph{$n$-dimensionale Kette} (\emph{$n$-Kette}) einer simplizialen
  Menge $X$ ist ein Element der freien abelschen Gruppe $C_n(X)$, welche von
  allen $n$-Simplizes erzeugt wird.
\end{definition}

