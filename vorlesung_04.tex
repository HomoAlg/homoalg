\begin{prop}
  Eine simpliziale Menge $\tilde X$ kann genau dann aus einem Verklebedatum $X$
  erhalten werden, wenn für jeden nicht-degenerierten Simplex $x \in \tilde
  X_n$ und für jede streng monotone Abbildung $f:[m]\to [n]$ der Simplex
  $\tilde X(f)(x)$ ebenfalls nicht-degeneriert ist. In diesem Fall ist $X$ im
  Wesentlichen durch $\tilde X$ bestimmt.
\end{prop}
\begin{proof}
  Übungsaufgabe.
\end{proof}

\subsubsection{Der klassifizierende Raum einer Gruppe}
Sei $G$ eine Gruppe. Setze $(BG)_n := G^n$ und für $f:[m]\to[n]$ monoton,
setzen wir $BG(f): G^n\to G^m,\ (g_1,\ldots,g_n)\mapsto (h_1,\ldots,h_m)$,
wobei
\[ h_i := \prod_{j = f(i-1)+1}^{f(i)} g_j\,.\]


\begin{beispiel}
  $f: [3]\to[4], 0\mapsto 0, 1,2\mapsto 2, 3\mapsto 4$ 
  \Bild
\end{beispiel}

\begin{definition}
  Die geometrische Realisierung $|BG|$ heißt der \emph{klassifizierende Raum
  von $G$}.
\end{definition}


\subsubsection{Nicht-ausgeartete Simplizes}
\label{subsub:nicht_ausgeartet}
\begin{definition}[ausgearteter Simplex]
  Sei $X$ eine simpliziale Menge. Ein $n$-Simplex $x \in X_n$ heißt 
  \emph{ausgeartet}, falls $x = X(f)(y)$ für $f:[n]\to[m]$ mit $m<n$,
  $y\in X_m$ und $f$ surjektiv.
\end{definition}

\begin{bem}
  Ist $x$ nicht-ausgeartet und $x = X(f)(y)$ für ein $f$ und $y$, so ist
  $f$ eine Injektion. Ansonsten zerlege $f = f_1\circ f_2$ mit $f_1$ injektiv
  und $f_2$ surjektiv. Damit $x = X(f_1\circ f_2)(y) = X(f_2)(X(f_1)(y))$.
\end{bem}

Ist $X$ eine simpliziale Menge, so sei 
$X\subk n := \{ x\in X_n \mid x\text{ nicht-ausgeartet}\}$ und
\begin{center}
  \begin{tikzcd}
    \tau: \coprod_{n\geq 0} \Delta_n \times X_n \rar & {} |X| \\
    \coprod_{n\geq 0} \mathring\Delta_n \times X\subk n
    \uar[hookrightarrow] \urar{\mathring\tau}
  \end{tikzcd}
\end{center}

\begin{prop}
  \label{prop:tauo_bijektion} 
  $\mathring\tau$ ist eine Bijektion.
\end{prop}

Um diese Proposition zu beweisen, brauchen wir einige Lemmas.

\begin{lemma}
  \label{lemma:ex_eind_paar}
  Für jedes $x \in X_n$ existiert ein eindeutiges Paar $(f,y)$, mit
  $y \in X\subk m$, $f:[n]\to[m]$ monoton, surjektiv, so dass $x = X(f)(y)$.
\end{lemma}
\begin{proof}
  Die Existenz folgt aus der Definition von $X\subk m$.
  Zur Eindeutigkeit: Seien $(f,y)$, $f:[n]\to[m]$ und $(f',y')$,
  $f':[n]\to [m']$ zwei solcher Paare mit $x = X(f)(y) = X(f')(y')$.
  Sei $g:[m]\to[n]$ ein monotoner Schnitt von $f$, d.h. 
  $f\circ g = \id_{[m]}$. Dann ist 
  \[ y = X(\id)(y) = X(f\circ g)(y) = X(g)X(f)(y) = X(g)(x). \]
  Damit $y = X(g)X(f')(y') = X(f'\circ g)(y')$.
  Da $y$ nicht-entartet ist, ist $f'\circ g:[m]\to [m']$ eine Injektion, so
  dass $m\leq m'$. Analog $m'\leq m$, also $m = m'$. Damit ist
  $f'\circ g$ eine monotone Injektion von $[m]$ nach $[m]$, also
  $f'\circ g = \id_{[m]}$, also $y = y'$. Da $f'\circ g = \id$ für alle
  Schnitte $g$ von $f$, folgt $f' = f$.
\end{proof}


\begin{folgerung}
  \label{folgerung:aufteilen_in_surj_inj}
  Sei $x \in X_n$ ein $n$-Simplex, $y\in X_m$ nicht-entartet, 
  $f:[n]\to[m]$ eine monotone Surjektion mit $x = X(f)(y)$. Sei weiter 
  $z \in X_l$ und $g: [n]\to[l]$ eine monotone Surjektion mit $x=X(g)(z)$. Dann
  faktorisiert $f$ als $f = h\circ g$ mit einem $h:[l]\to[m]$, so dass
  $z = X(h)(y)$.
\end{folgerung}
\begin{proof}
  Sei $(h',y')$ das Paar aus \cref{lemma:ex_eind_paar} mit $z = X(h')(y')$.
  Dann ist $x = X(g)(z) = X(g)(X(h')(y')) = X(h'\circ g)(y')$. Damit ist
  $h'\circ g = f$ und $y = y'$. Setze $h:= h'$.
\end{proof}

\paragraph{Beweis von \cref{prop:tauo_bijektion}}

\begin{proof}[von \cref{prop:tauo_bijektion}]
  \begin{enumerate}
    \item $\mathring\tau$ ist surjektiv: Sei $p\in |X|$. Sei
      $k$ die kleinste Dimension, so dass ein $(s,x) \in\Delta_k\times X_k$ mit
      $\tau(s,x) = p$ existiert. Wir wollen nun zeigen, dass
      $(s,x) \in \mathring\Delta_k \times X\subk k$: Ist $k=0$, so ist 
      $\Delta_0 = \mathring\Delta_0$ und $X\subk 0 = X_0$, also nichts zu
      zeigen. Ist $x$ uasgeartet, so ist $x = X(f)(y)$, $f:[k]\to[l]$ monoton,
      $l < k$. Dann ist $\tau(s,x) = \tau(\Delta_f(s),y)$. Widerspruch zur
      Minimalität von $k$. Also $x \in X\subk k$. Ist
      $s\notin\mathring\Delta_k$, so existiert ein injektives $f:[l]\to[k]$
      monoton mit $s = \Delta_f(t)$ für $t\in \Delta_l$. Damit 
      $\tau(s,x) = \tau(\Delta_f(t),x) = \tau(t,X(f)(x))$ im Widerspruch zu 
      $l<k$.
    \item $\mathring\tau$ ist injektiv: Seien 
      $(s,x) \in \mathring\Delta_k \times X\subk k$,
      $(s',x')\in\mathring\Delta_l\times X\subk l$ mit $\tau(s,x) =
      \tau(s',x') (\ast)$, so ist zu zeigen, dass $s = s'$ und $x = x'$.
      Wegen $(\ast)$ existiert 
      \[ (s,x) = (s_0,x_0) \sim (s_1,x_1) \sim \ldots\sim (s_N,x_N) = (s',x')\]
      wobei $\sim$ für $(s_i,x_i) \overset{f_i^+}{\mapsto} (s_{i+1},x_{i+1})$
      oder $(s_i,x_i) \overset{f_i^-}{\leftmapsto} (s_{i+1},x_{i+1})$ 
      steht. Ohne Einschränkung können wir ``Zickzack'' annehmen.
      Wir zeigen: Ist $N=1$, so ist $f_0 = \id$ und ist $N>1$, so existiert
      eine kürzere Kette.
    \item Beh: Da $(s,x)\in\mathring\Delta_k \times X\subk k$, so ist
      $f_0^+$ eine Injektion und $f_0^-$ eine Surjektion.
      Im Fall $f_0^+$ ist $x = X(f_0^+)(x_1)$. Dann ist $f_0^+$ eine Injektion
      nach \autoref{subsub:nicht_ausgeartet}. Im Fall $f_0^-$ ist
      $s = \Delta_{f_0^-}(s_1)$. Da $s \in \mathring\Delta_k$, muss $f_0^-$
      surjektiv sein. Wir folgern: $N=1$.\autoref{subsub:nicht_ausgeartet}. Im
      Fall $f_0^-$ sit
      $s = \Delta_{f_0^-}(s_1)$. Da $s \in \mathring\Delta_k$, muss $f_0^-$
      surjektiv sein. Wir folgern: Für $N=1$ ist $f_0^- / f_0^+$ ist injektiv
      und surjektiv, also gleich $\id$.
    \item Für $N\geq 2$ betrachte
      \begin{center}
        \begin{tikzcd}
          (s_i,x_i) \rar[mapsto]{f_i^+} & (s_{i+1},x_{i+1}) &
            (s_{i+2},x_{i+2}) \lar[mapsto]{f_{i+1}^-}\\[5pt]
          {}[m_i] \rar &{} [m_{i+1}] &{} [m_{i+2}] \lar
        \end{tikzcd}
      \end{center}
      Sei außerdem $f_i^+$ injektiv. Wir zeigen, dass dann folgende Kette
      existiert:
      \begin{center}
        \begin{tikzcd}
          (s_i,x_i) & (s_{i+1},x_{i+1}) \lar[mapsto]{g} \rar[mapsto]{h}&
            (s_{i+2},x_{i+2})\\
          {}[m_i] \rar &{} [m_{i+1}] &{} [m_{i+2}] \lar
        \end{tikzcd}
      \end{center}
      Sei $ I:=(f_{i+1}^-)\inv(f_i^+([m_i]))$, $l:= |I| -1$. Sei 
      $h:[l]\to[m_{i+2}]$ die injektive monotone Abbildung mit Bild $I$
      Dann existiert genau ein $g:[l]\to[m_i]$ mit $f_i^+\circ g = 
      f_{i+1}^- \circ h$. Da $\im h=I$, ist 
      $\im\Delta_h = (\Delta_{f_{i+1}^-)\inv(\Delta_{f_i^-}(\Delta_{m_i}))$.
      Da $\Delta_{f_{i+1}^-}(s_{i+2}) = s_{i+1} = \Delta_{f_i^+}(s_i)$,
      existiert ein $t\in \Delta_l$ mit $\Delta_h(t) = s_{i+2}$. Außerdem:
      \[ \Delta_{f_i^+}(\Delta_g(t)) = \Delta_{f_{i+1}^-}(\Delta_h(t))
        = s_{i+1} = \Delta_{f_i^+}(s_i)\]
      Da $\Delta_{f_i^+}$ injektiv, folgt $\Delta_g(t) = s_i$. Setze
      $y := X(h)(x_{i+2}) = X(g)(x_i)$.
    \item Sei $N\geq 2$, so beginnt die Kette mit 
      $(s,x) \overset{f_0^+}{\mapsto} (s_1,x_1) \overset{f_1^-}\leftmapsto
      (s_2,x_2)$, so ist $f_0^+$ nach (4) eine Injektion. Nach (3) erhalten wir
      dann eine Kette der Länge 2 (für $N=2$ oder $N-1$ für $N\geq 3$) der Form
      \[ (s_0,x_0) \overset{f_0^-}{\leftmapsto} (s_1,x_1) 
        \overset{f_1^+}{\mapsto} (s_2,x_2)\]
      Nach (4) ist $f_0^-$ eine Surjektion.
    \item Ist $N = 2$, also $(s_2,x_2) \in \mathring\Delta_l\times X\subk l$,
      so ist $f_1^+$ surjektiv nach (3).
      Da $x_0,x_2$ nicht-ausgeartet, folgt nach \cref{lemma:ex_eind_paar}, dass
      $f_0^- = f_1^+$ und $x_0 = x_2$, also $s_0 = s_2$.
    \item Sei also $N\geq 3$. Schreibe $f_1^+ = i\circ p$, wobei 
      $p:[m_1] \to[l]$ monotone Surjektion und $i:[l]\to[m_2]$ monotone
      Injektion. Da $x_0$ nicht-ausgeartet und $f_0^-$ surjektiv, existiert
      nach \cref{folgerung:aufteilen_in_surj_inj} 
      eine Faktorisierung $f_0^- = g \circ p$ für ein 
      $g:[l]\to[k]$. Damit ersetzen wir 
      $(s_0,x_0)\overset{f_0^-}{\leftmapsto} (s_1,x_1) \overset{f_1^+}{\mapsto}
      (s_2,x_2)$ durch
      \[ (s_0,x_0) \overset{g}{\leftmapsto} (\Delta_p(s_1, X(g)(x_0))
        \overset{i}{\mapsto} (s_2,x_2)\,,\]
      d.h. ohne Einschränkung ist $f_1^+$ eine Injektion. Nach (4) können wir
      also $(s_1,x_1) \overset{f_1^+}{\mapsto} (s_2,x_2) 
      \overset{f_2^-}{\leftmapsto} (s_3,x_3)$ durch
      $(s_1,x_1) \overset{f_1^-}{\leftmapsto} (s_2,x_2)
      \overset{f_2^+}{\mapsto} (s_3,x_3)$ ersetzen und wir erhalten eine Kette
      der Länge $N-2$.
  \end{enumerate}
\end{proof}

\begin{folgerung}
  Sei $|X|$ ein triangulierter Raum mit Verklebedatum $(X\subk n, X(f))$. Sei
  $\tilde X$ die dazugehörige simpliziale Menge Dann $|\tilde X| = |X|$.
\end{folgerung}
\begin{proof}
  \makebox{}
  \begin{center}
    \begin{tikzcd}
      \coprod_{n\geq 0} \Delta_n\times \tilde X_n \dar
        \rar{(s,\tilde x) = (s,g) \mapsto (\Delta_g(s),x)}
        &[4cm] \coprod_{n\geq 0} \Delta_n \times X\subk n \dar
        \\ {}
      |\tilde X| \rar[dashed]{\varphi}
        &{} |X|
    \end{tikzcd}
  \end{center}
  Dann ist nach Übungsaufgabe $\varphi$ ein Homöomorphismus.
\end{proof}
